\documentclass[12pt]{report}

\usepackage[italian]{babel}
\usepackage[hidelinks]{hyperref}
\usepackage[htt]{hyphenat}


% Title Page
\title{Povinator3000 \\

{\Large Manuale Utente}}
\author{Daniele Parmeggiani}
\date{\today}


\begin{document}
\pagenumbering{gobble}
\maketitle
\pagenumbering{arabic}
\tableofcontents

\chapter{Guida all'Utilizzo}

Il codice sorgente di Povinator3000 si trova su GitHub \url{https://github.com/dpdani/povinator3000}.

Nel manuale si farà spesso riferimento a URL come \url{http://localhost:3000/} per indicare il percorso di Povinator.

Nel caso tale URL non rispondesse, chiedere all'installatore a quale percorso Povinator sia raggiungibile.

\section{Preparazione al caricamento delle
Presentazioni}\label{preparazione-al-caricamento-delle-presentazioni}

Per permettere agli studenti di caricare le proprie presentazioni, è necessario preparare una cartella su Google Drive che ospiterà i file degli studenti.

\subsection{Struttura della Cartella Lauree}\label{struttura-della-cartella-lauree}

Povinator necessita per funzionare di una struttura ben definita delle cartelle che ospitano le presentazioni dei candidati.

È necessario costruire la cartella ``Lauree'' in modo che contenga nel primo livello una cartella per ognuno dei Dipartimenti che parteciperanno alla sessione di laurea.
Al secondo livello, per ogni Dipartimento è necessario creare una sotto-cartella per ogni Commissione di Laurea di tale Dipartimento.

Per esempio, si potrebbe ottenere una struttura simile alla seguente:

\begin{samepage}
\begin{verbatim}
+ Lauree 2020-04-26
  + CIBIO
    + BCM 1
    + BCM 2
  + DISI
    + Commissione 1
    + Commissione 2
\end{verbatim}
\end{samepage}

Suggerimento: nominare le cartelle con anno-mese-giorno permette di far coincidere l'ordine alfabetico con quello cronologico.

\subsection{Generazione del Form}\label{generazione-del-form}

Povinator3000 richiede che il Google Form che raccoglierà le presentazioni sia costruito in un certo modo.
Al fine di produrlo correttamente, è necessario compiere i seguenti passaggi:

\begin{enumerate}
\item Impostare la collezione di email, la richiesta di essere loggati e di limitazione ad una risposta.
\item Aggiungere, \emph{in questo ordine}, i seguenti campi ognuno dei quali dev'essere necessariamente inserito per inviare la risposta:
    \begin{enumerate}
    \item Nome
    \item Cognome
    \item Data della presentazione
    \item Livello di Laurea (se triennale o magistrale)
    \item Dipartimento / Facoltà
    \end{enumerate}
\item Per ognuno dei Dipartimenti, creare una nuova pagina (\emph{page break}) e inserire un campo obbligatorio contenente ognuna delle Commissioni istituite.
\item Creare una nuova pagina e inserire un campo per il caricamento di file.
\end{enumerate}

Nelle impostazioni, selezionare la generazione di uno Sheet che conterrà le risposte.

%Dopo aver strutturato la cartella lauree come indicato nella sezione precedente, navigare su Google Drive in tale cartella e copiare l'URL presente nella barra di navigazione del proprio browser.
%
%L'URL sarà simile a:
%\texttt{https://drive.google.com/drive/u/0/folders\-/0BwwA4oUTeiV1TGRPeTVjaWRDY1E}.
%
%Navigare ora su \url{http://localhost:3000/form} e incollare il link appena copiato nella apposita casella e procedere.

Dovrebbe essere fornito un Form di esempio che è possibile copiare e riutilizzare.


\section{Download delle Presentazioni}\label{download-delle-presentazioni}

Una volta che viene chiuso il Form e gli studenti non hanno più la possibilità di caricare le proprie presentazioni, si può usare Povinator per procedere con la routine di rinominazione, spostamento e download delle presentazioni.

\subsection{Pagina Principale}\label{pagina-principale}

Per utilizzare Povinator, navigare con il proprio browser su \url{http://localhost:3000/}.

Quella che viene mostrata è la pagina iniziale di Povinator, la quale mette a disposizione un pulsante per iniziare la procedura di Download delle Presentazioni.
Dalla pagina che viene aperta quando si preme sul pulsante, è possibile indicare l'URL della cartella di Google Drive su cui operare.

L'URL da inserire sarà simile a:
\texttt{https://drive.google.com/drive/u/\-0/folders/0BwwA4oUTeiV1TGRPeTVjaWRDY1E}.
In particolare, questo URL può essere trovato navigando su Google Drive nella cartella da utilizzare e copiando l'URL nella barra di navigazione del proprio browser.

%Nella stessa pagina è anche possibile selezionare l'opzione di scaricare nel computer locale i file delle presentazioni che sono stati caricati dagli studenti, una volta che Povinator li abbia rinominati e spostati nella cartella corretta, all'interno di Drive.

Nota: eseguire più volte Povinator sugli stessi file non produrrà effetti indesiderati.

\subsection{Fogli di Risposte}\label{fogli-di-risposte}

Una volta inserito l'URL e selezionate le opzioni desiderate, è possibile procedere alla schermata successiva, premendo il pulsante con la freccia.

In questa schermata è possibile selezionare quali file siano i fogli elettronici contenenti le risposte al Form presentato ai laureandi.

Povinator utilizzerà questi file per decidere come rinominare e spostare i file all'interno dell'archivio Drive.

È necessario selezionare almeno un file perché Povinator riesca ad effettuare le consuete operazioni. Nel caso non venga selezionato nulla, Povinator procederà senza effettuare nessuna operazione.

Povinator mette a disposizione dell'utente per la selezione \emph{solo} i file che rispettano le caratteristiche adatte, ovvero

\begin{itemize}
\item
  il file deve essere un Google Spreadsheet;
\item
  il nome del file deve terminare con ``(Responses)'' o ``(Risposte)'' i
  quali sono i suffissi che Google applica ai file di risposta dei
  Google Form.
\end{itemize}

Nel caso Povinator non riesca a trovare nessun file che rispetti queste caratteristiche, anziché mostrare la normale pagina di selezione, mostrerà una schermata di errore.

\subsection{Pagina Finale}\label{pagina-finale}

Dopo aver selezionato i Fogli di Risposta, Povinator procederà a rinominare e spostare i file caricati dagli studenti nelle cartelle corrette.

I file verranno rinominati con il nome e cognome dello studente che li ha caricati, così come lo stesso studente avrà indicato compilando il Form.

In questa pagina, sarà possibile prendere visione di tutte le operazioni compiute da Povinator, in forma testuale.

In calce al corpo della pagina, Povinator mostrerà i percorsi in cui sono stati scaricati i file e dove è stato creato l'archivio ZIP.
Da qui è possibile scaricare il file ZIP.


\chapter{Installazione}\label{installazione}

\section{Python}\label{python}

Come prima cosa assicurarsi che Python sia disponibile sul proprio
sistema in versione \(\geq\) 3.8.

Per Windows è possibile scaricarlo da
\url{https://www.python.org/downloads/windows/}.

Per Linux, consultare le modalità di installazione della propria
distribuzione. Inoltre, è assai probabile che sia già installato. Si può
verificare inserendo il comando da terminale:

\begin{verbatim}
$ python3 --version
Python 3.8.6
\end{verbatim}

Se l'esecuzione di tale comando restituisce un output simile a quello
mostrato, allora Python è stato installato con successo.

\section{Git}\label{git}

Successivamente, è necessario installare Git.

Per Windows: \url{https://git-scm.com/download/win}.

Nuovamente, per Linux consultare le modalità specifiche della propria distribuzione.
Anche Git è probabile che sia preinstallato.

Verificare la corretta installazione:

\begin{verbatim}
$ git --version
git version 2.25.2
\end{verbatim}

\section{Povinator3000}\label{povinator3000}

Per installare tutte le componenti necessarie al funzionamento di Povinator è sufficiente eseguire:

\begin{verbatim}
$ git clone https://github.com/dpdani/povinator3000.git
$ cd povinator3000
$ make install
\end{verbatim}

Se non si ha la possibilità di utilizzare il sistema make, è possibile ottenere il medesimo risultato eseguendo:

\begin{verbatim}
$ git clone https://github.com/dpdani/povinator3000.git
$ cd povinator3000
$ python3 -m venv venv
$ source venv/bin/activate  # per linux
$ venv\bin\activate.bat  # per windows
$ pip install -r requirements.txt
$ python -m povinator3000
\end{verbatim}

Verificare la corretta installazione navigando su
\url{http://localhost:3000/}.

Se sul browser viene mostrata la pagina di benvenuto di Povinator,
allora l'installazione è avvenuta con successo.

\section{Avviare Povinator}

Per avviare Povinator, è sufficiente eseguire:

\begin{verbatim}
$ make start
\end{verbatim}

Questo avvierà Povinator sottintendendo che esso dovrà essere raggiungibile su \texttt{localhost:3000}.
Se si desidera usare un altro \emph{bind}, è sufficiente passarlo come argomento:

\begin{verbatim}
$ make start bind=192.168.1.1:80
\end{verbatim}


\section{Aggiornamenti}

Se il repository di GitHub dispone di una nuova versione di Povinator, per aggiornare il software, sarà necessario eseguire il seguente comando:

\begin{verbatim}
$ make update
\end{verbatim}

Con il seguente comando invece, si può riattivare Povinator dopo l'aggiornamento (questo eseguirà anche \texttt{make update}):

\begin{verbatim}
$ make upgrade
\end{verbatim}

\end{document}          
